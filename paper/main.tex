\documentclass[11pt, letter]{article}

\usepackage[margin=1in]{geometry}
\usepackage{amsmath}
\usepackage{amsthm}
\usepackage{amssymb}
\usepackage{enumitem}

\newcommand{\piece}[2]{\mathbf{#1}\text{#2}}

\theoremstyle{definition}
\newtheorem{alg}{Algorithm}

\title{Training a Linear Heuristic on Chess Positions for the Minimax Algorithm}
\author{Tae Hyung Kim}

\begin{document}

\maketitle

\section{Introduction}
First, we make clear the notation that we shall use. We shall represent the type
of chess pieces by the following abbreviations:
\[\{ \mathbf{K} \leftrightarrow \text{King}, 
\mathbf{Q} \leftrightarrow \text{Queen},
\mathbf{R} \leftrightarrow \text{Rook},
\mathbf{B} \leftrightarrow \text{Bishop},
\mathbf{N} \leftrightarrow \text{Knight},
\mathbf{P} \leftrightarrow \text{Pawn}
\}. \]
In addition, we shall use the standard method of identifying location of a piece. 
That is, each column shall be lettered $\text{a}, \dots, \text{f}$ while each row
will be numbered $1, \dots, 8$. Each position will be written with the column
letter and the row number consecutively; for example, we would say that the
white king initially occupies the position $\text{a}4$. Finally, we represent a
piece, which includes its position, by the piece type followed by its position.
For example, the king is denoted by $\piece{K}{a4}$. 

Recall the algorithm of minimax with alpha-beta pruning. Suppose that we 
represent a \texttt{Board} as an object (assume Python for simplicity) with the
method \texttt{get\_moves}. In addition, suppose that we have a heuristic
function $h : \mathtt{Board} \to \mathbb R$ which evaluates how good a board is.
\renewcommand*{\thealg}{M}
\begin{alg}
    Suppose we have a board \texttt{board}.
    \begin{enumerate}[label=\textbf{\thealg\arabic*}.]
        \item If 
    \end{enumerate}
\end{alg}

Define variables $\theta_{P_1, P_2}$ where $P_1, P_2$ are any two pieces. 
Now, the heuristic of interest in this paper is $h_\theta : \mathtt{Board} \to
\mathbb{R}$ defined by
\[ h_\theta(\texttt{board}) = \sum_{P_1, P_2 \in \texttt{board}} \theta_{P_1, P_2} \cdot \chi_\texttt{board}(P_1) \chi_\texttt{board}(P_2) \]
where $\chi_\texttt{board} : \texttt{Piece} \to \{0, 1\}$ is the indicator of
whether a certain piece is in the board or not. Note that the product 
$\chi_\texttt{board}(P_1) \chi_\texttt{board}(P_2)$ simply returns $1$ if the
pair of pieces $P_1, P_2$ is in the board, and $0$ otherwise. 

The ultimate goal is to find the $\theta$ which will provide the best heuristic
for the minimax algorithm. Given that the data we will be given is whether a
certain board is ``good'' or ``bad'', the problem of computing $\theta$ is a
logistic regression problem, as we wish to minimize the cost $\log(g(h_\theta(\texttt{board})$ where $g$ is the sigmoid function. 

\end{document}
